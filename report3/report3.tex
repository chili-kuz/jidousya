\documentclass[a4paper,12pt]{jarticle}
\usepackage[dvipdfmx]{graphicx}
\usepackage{amsmath}
\usepackage{subfigure}
\usepackage{comment}
\usepackage{array}


\setlength{\hoffset}{0cm}
\setlength{\oddsidemargin}{-3mm}
\setlength{\evensidemargin}{-3cm}
\setlength{\marginparsep}{0cm}
\setlength{\marginparwidth}{0cm}
\setlength{\textheight}{24.7cm}
\setlength{\textwidth}{17cm}
\setlength{\topmargin}{-45pt}

\renewcommand{\baselinestretch}{1.6}
\renewcommand{\floatpagefraction}{1}
\renewcommand{\topfraction}{1}
\renewcommand{\bottomfraction}{1}
\renewcommand{\textfraction}{0}
\renewcommand{\labelenumi}{(\arabic{enumi})}
%\renewcommand{\figurename}{Fig.} %図をFig.にする


%図のキャプションからコロン:を消す
\makeatletter
\long\def\@makecaption#1#2{% #1=図表番号、#2=キャプション本文
\sbox\@tempboxa{#1. #2}
\ifdim \wd\@tempboxa >\hsize
#1 #2\par 
\else
\hb@xt@\hsize{\hfil\box\@tempboxa\hfil}
\fi}
\makeatother



\begin{document}
%
\title{\vspace{-30mm}  自動車工学特論~~レポート \\ 機械知能工学専攻~~16344217~~津上~祐典}
\date{}
%
\maketitle
%
\vspace{-20mm}
%\parindent = 0pt %すべての段落で字下げをしない
%
バッテリーEV(BEV)の利点は,エネルギー消費や環境に及ぼす影響が少ない点
で優れているが,普及が進んでいない.
%
その理由として二つ考えられる.

一つ目は充電所の普及が進んでいないからであると考えられる.
%
私はバイクを持っており,度々ツーリングや旅行に行くことがある.
%
ガソリンがなくなりそうになったら,当然ガソリンを入れるため携帯のナビでガソリン
スタンドを探す.
%
よっぽど山奥でない限りはガソリンスタンドは簡単に見つかる.
%
しかし,電気自動車用の充電所はほとんど見たことがない.
%
近所では,とある立体駐車場の小さなスペースに二台分あるだけである.
%
私が,電気自動車を持ってないせいなのか充電所が世の中に全くないイメージがある.
%
となると,車を購入する際に電気自動車を勧められ,エレルギー消費が良くて
も購入しにくい.
%
したがって,充電所を多く増やし,積極的に世間の人にアピールするべきだと考えられる.
%
そうすることでBEVの普及が進むのではないかと考えられる.
%
具体的には,TVのCMや広告,もしくは電力会社と連携していくことが挙げられる.

二つ目は給油(充電)時間の長さ,走行距離の短さであると考えられる.
%
普通に充電した場合は,FULLになるまで数時間以上かかる場合がある.
%
また,比較的料金の安い夜間に充電するため自宅にBEVを駐車するスペースや充電設備が必須と
なってくる.
%
このようになるとBEVに対する初期投資額が大きくなると考えられる.
%
また,エネルギー消費は優れているがバッテリーに充電できる電力が少なく,ガ
ソリン車に比べて走行距離が短い.
%
これらを解決するためには,急速充電施設の普及や家庭でも急速充電出来る設備
を導入しやすくする必要があると考えられる.
%
また,小型で大容量のバッテリーを開発することで,少なくともガソリン車以上
の走行距離を出せるようにするべきと考えられる.
%
充電池は使用方法を間違えると大きな事故を起こしかねないので,そういった知
識を身につける必要があると考えられる.

まとめとして,BEVの性能を上げるでなく,まずはじめに充電所などのインフラ
整備を充実させること,また,人々にBEVについて知ってもらうことから始める
必要があると考えられる.
\end{document}
