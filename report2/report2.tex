\documentclass[a4paper,12pt]{jarticle}
\usepackage[dvipdfmx]{graphicx}
\usepackage{amsmath}
\usepackage{subfigure}
\usepackage{comment}
\usepackage{array}


\setlength{\hoffset}{0cm}
\setlength{\oddsidemargin}{-3mm}
\setlength{\evensidemargin}{-3cm}
\setlength{\marginparsep}{0cm}
\setlength{\marginparwidth}{0cm}
\setlength{\textheight}{24.7cm}
\setlength{\textwidth}{17cm}
\setlength{\topmargin}{-45pt}

\renewcommand{\baselinestretch}{1.6}
\renewcommand{\floatpagefraction}{1}
\renewcommand{\topfraction}{1}
\renewcommand{\bottomfraction}{1}
\renewcommand{\textfraction}{0}
\renewcommand{\labelenumi}{(\arabic{enumi})}
%\renewcommand{\figurename}{Fig.} %図をFig.にする


%図のキャプションからコロン:を消す
\makeatletter
\long\def\@makecaption#1#2{% #1=図表番号、#2=キャプション本文
\sbox\@tempboxa{#1. #2}
\ifdim \wd\@tempboxa >\hsize
#1 #2\par 
\else
\hb@xt@\hsize{\hfil\box\@tempboxa\hfil}
\fi}
\makeatother



\begin{document}
%
\title{\vspace{-30mm}  自動車工学特論~~レポート \\ 機械知能工学専攻~~16344217~~津上~祐典}
\date{}
%
\maketitle
%
\vspace{-20mm}
%\parindent = 0pt %すべての段落で字下げをしない
%
パワートレーンの課題としてエネルギー対応や二酸化炭素の削減,大気汚染防止
が挙げられる.将来の自動車用のパワートレーンは省エネルギー,かつ二酸化炭
素を放出せず環境に優しいものであるべきと思う.そのためには,水素もしくは
電気で動作するようなパワートレーンの開発が必要と考えられる.その技術の課
題として電気の場合,一回の充電での最高走行距離が短い,また補充に多くの
時間を要することがあげられる.そ
の課題を解決するためには,小さくても多く電力を蓄えることが出来るバッテリー
の開発,また,省電力で多く走行できるパワートレーンの機構の開発が必要である
と考えられる.また,急速に充電するシステムも必要だと考えられる.更に開発が出来た場合,電力補給場所(ガソリンスタンドの電気
版みたいな)の普及が必要になってくると思われる.

\end{document}
